%setup
\documentclass[]{article}
\usepackage[margin=1in]{geometry}
\usepackage{amsmath}
\usepackage{amssymb}
\usepackage{physics}
\usepackage{fancyhdr}
\usepackage{pgfplots}

%settings
\pgfplotsset{compat=1.16}
\pagestyle{fancy}

%opening
\title{SPH4U6 Review: Energy}
\author{Ryan Rambali}

%header + footer
\renewcommand{\headrulewidth}{0pt}
\renewcommand{\footrulewidth}{0pt}
\fancyhf{}
\rhead{
	Ryan Rambali\\
	Energy Review\\
	SPH4U6\\
}
\rfoot{Page \thepage}

\begin{document}

\subsection*{Page 214-218}
\paragraph{4.}
	\textit{
		Two identical marbles are dropped in a classroom.
		Marble A is dropped from 1.00 m, and marble B is
		dropped from 0.50 m. Compare the kinetic energies
		of the two marbles just before they strike the ground.
	}\\	
\par
\paragraph{5.}
\textit{
 		A 0.30 kg soccer ball is released from the top of a
		10 m building. The ball strikes the ground with a
		speed of 12 m/s. Use the conservation of energy to
		determine the energy lost due to the work done by
		air resistance.
}\\	
\paragraph{8.}
\textit{
		All moving objects have kinetic energy.
}\\	
\par
\paragraph{10.}
\textit{
	The gravitational potential energy of an object 5 m
	above the ground in Ontario is the same as an identical
	object 5 m above the ground on the Moon.
}\\	
\par
\paragraph{11.}
\textit{
	The joule (J) is the SI unit for three quantities: work,
	energy, and power.
}\\	
\par
\paragraph{12.}
\textit{
	A marble is shot from a slingshot on a planet with no
	atmosphere. At any given moment, before the marble
	hits the ground, the sum of the kinetic energy and the
	gravitational potential energy is constant.
}\\	
\par
\paragraph{13.}
\textit{
	The farther you pull a spring beyond its equilibrium
	point, the more work you do on it.
}\\	
\par
\paragraph{14.}
\textit{
	In an oscillating spring, the elastic potential energy
	when the spring is completely compressed is equal to
	the kinetic energy when the spring is fully extended
}\\	
\par
\paragraph{18.}
\textit{
	 A car is parked on a hill. The gravitational force on the
	car is 9.31 3 103 N straight downward, and the angle of
	the hill is 4.00° from the horizontal (Figure 3). The car’s
	brakes fail, and the car slides 30.0 m downhill. 
}\\	
\par
\paragraph{21.}
\textit{
	A sprinter with a mass of 68 kg is running at a
	speed of 5.8 m/s. In a burst of speed to win the race,
	she increases her speed to 6.9 m/s. Determine the
	work that the sprinter does to increase her speed.
}\\	
\par
\paragraph{22.}
\textit{
	In a curling match, a 20.0 kg stone (Figure 4) with an
	initial speed of 2.0 m/s glides to a stop after 30.0 m.
	Determine the work done on the stone by friction
}\\	
\par
\paragraph{54.}
\textit{
	An amusement park uses large compressible springs
	to stop cars at the end of a ride. Assume the springs
	are ideal, with no weight, mass, or damping losses.
	The combined mass of the car and passengers
	averages 450 kg, and the cars make contact with the
	spring at a speed of 3.5 m/s. Determine the minimum
	spring constant to bring the car and its riders to a
	stop in 2.0 m.
}\\	
\par
\paragraph{57.}
\textit{
	Suppose you set a spring with spring constant
	4.5 N/m into damped harmonic motion at noon,
	measuring its maximum displacement from
	equilibrium to be 0.75 m. When you return 15 min
	later, the spring is still oscillating, but its maximum
	displacement has decreased to 0.5 m.
}\\\\	
	\indent\textit{
	\textbf{A.} Determine how much energy the system
	has lost
	}\\\\	
	\indent\textit{
	\textbf{B.} What is the power loss of the system?
	}\\	
\par
\paragraph{58.}
\textit{
	A ball is attached to a vertical spring with a spring
	constant of 6.0 N/m. It is held at the equilibrium
	position of the spring and then released. It falls
	0.40 m and then bounces back up again. Calculate
	the mass of the ball.
}\\	
\par
\paragraph{59.}
\textit{
	A ball of mass 0.50 kg is attached to a horizontal
	spring. The spring is compressed 0.25 m from its
	equilibrium and then released. The ball undergoes
	simple harmonic motion, achieving a maximum
	speed of 1.5 m/s. 
}\\\\	
	\indent\textit{
	\textbf{A.} Determine the spring constant.
	}\\\\
	\indent\textit{
	\textbf{B.} Calculate the speed of the ball when the spring is halfway to its equilibrium point.
	}\\\\
	\indent\textit{
	\textbf{C.} When the ball is halfway to its equilibrium point, what fraction of its energy has been converted\\
	\indent\indent from elastic potential energy to kinetic energy?
	}\\	
\par
\paragraph{60.}
\textit{
	Two objects of different masses are suspended from
	two springs that have the same spring constant. The
	heavier of the two objects will extend its spring farther
	beyond the equilibrium point. Why? 
}\\	
\par
\end{document}
