%setup
\documentclass[]{article}
\usepackage[margin=1in]{geometry}
\usepackage{amsmath}
\usepackage{amssymb}
\usepackage{physics}
\usepackage{fancyhdr}
\usepackage{pgfplots}

%settings
\pgfplotsset{compat=1.16}
\pagestyle{fancy}

%opening
\title{SPH4U6 Review: Momentum}
\author{Ryan Rambali}

%header + footer
\renewcommand{\headrulewidth}{0pt}
\renewcommand{\footrulewidth}{0pt}
\fancyhf{}
\rhead{
	Ryan Rambali\\
	Momentum Review\\
	SPH4U6\\
}
\rfoot{Page \thepage}

\begin{document}
	
	\subsection*{Page 264-269}
	\paragraph{1.}
	\textit{
		Suppose object A has greater momentum than object B.
		Which of the following can you conclude?
	}\\	
	\par
	\paragraph{2.}
	\textit{
		trampoline. Her speed as she hits the trampoline is
		5.2 m/s, and she comes to a stop in 0.20 s. What is
		the average magnitude of the force exerted on the
		gymnast by the trampoline?
	}\\	
	\paragraph{6.}
	\textit{
		When an acorn falls and hits the ground, Earth’s
		response is imperceptible. 
	}\\	
	\par
	\paragraph{7.}
	\textit{
		In an inelastic collision only momentum is
		conserved.
	}\\	
	\par
	\paragraph{8.}
	\textit{
		When two objects undergo a perfectly elastic head-on
		collision, each object will always have a final velocity
		equal to the initial velocity of the other object. 
	}\\	
	\par
	\paragraph{9.}
	\textit{
		A head-on collision is a collision in which the initial
		and the final velocities of colliding masses lie in the
		same line.
	}\\	
	\par
	\paragraph{12.}
	\textit{
		Verify, using the definition of momentum, that the
		units for momentum are the same as those for force
		multiplied by time.
	}\\	
	\par
	\paragraph{14.}
	\textit{
		Two construction workers use different hammers
		to pound in nails. Both swing their hammers with
		the same speed, and the duration of both hammers’
		collisions with the nails is equal. However, one worker
		seems to achieve more force than the other. Offer a
		possible explanation
	}\\	
	\par
	\paragraph{21.}
	\textit{
		Two tennis balls of equal mass are moving in
		directions opposite to each other. The tennis balls
		are travelling with equal speed when they collide
		head-on. You can assume that this collision is
		perfectly elastic. Describe in your own words what
		happens after the tennis balls collide. 
	}\\	
	\par
	\paragraph{23.}
	\textit{
		Two soccer players collide head-on and are stopped.
		If the mass of one player is 1.2 times the mass of the
		other player, what can you conclude about their initial
		speeds?
	}\\	
	\par
	\paragraph{22.}
	\textit{
		In a curling match, a 20.0 kg stone (Figure 4) with an
		initial speed of 2.0 m/s glides to a stop after 30.0 m.
		Determine the work done on the stone by friction
	}\\	
	\par
	\paragraph{26.}
	\textit{
		Hockey player 1 is travelling at a velocity of 12 m/s [N]
		and hockey player 2 is travelling at a velocity of
		18 m/s [S] when they collide head-on. After colliding,
		the hockey players hang on to each other and slide
		along the ice with a velocity of 4.0 m/s [S]. If hockey
		player 1 weighs 120 kg, calculate how much hockey
		player 2 weighs.
	}\\	
	\par
	\paragraph{28.}
	\textit{
		A 1.2 kg cart slides eastward down a frictionless
		ramp from a height of 1.8 m and then onto a
		horizontal surface where it has a head-on elastic
		collision with a stationary 2.0 kg cart cushioned
		by an ideal Hooke’s law spring. The maximum
		compression of the spring during the collision is
		2.0 cm.
	}\\\\	
	\indent\textit{
		\textbf{A.} Determine the spring constant
	}\\\\	
	\indent\textit{
		\textbf{B.} Calculate the velocity of each cart just after the collision.
	}\\\\	
	\indent\textit{
	\textbf{C.} After the collision, the 1.2 kg cart rebounds
	up the ramp. Determine the maximum height\\
	\indent\indent reached by the cart.
	}\\	
	\par
	\paragraph{31.}
	\textit{
		Two balls of different masses and equal speeds
		undergo a head-on elastic collision. If the balls are
		moving in opposite directions after the collision,
		how can you determine from the outcome of the
		collision which object has a greater mass?
	}\\	
	\par
	\paragraph{32.}
	\textit{
		A curling stone travelling at 5.0 m/s collides
		with a stationary stone of the same mass.
		Following the collision, the two stones travel
		at angles of 17° and 38° in opposite directions
		with respect to the initial motion of the first
		stone. 
	}\\\\	
	\indent\textit{
		\textbf{A.} Draw a diagram of the stones’ motion. 
	}\\\\
	\indent\textit{
		\textbf{B.} Calculate the speed of each stone after the
		collision
	}\\
	\par
	\paragraph{33.}
	\textit{
		During a spacewalk, three astronauts wearing
		jetpacks approach each other at equal speeds along
		lines equally spaced by an angle of 120° (Figure 1).
		As the astronauts approach each other, they take each
		other’s hands. If the astronauts come to rest after
		colliding, what conclusion can you draw?
	}\\	
	\par
	\paragraph{37.}
	\textit{
		Ball 1 of mass 0.1 kg makes an elastic head-on
		collision with ball 2 of unknown mass that is initially
		at rest. If ball 1 rebounds at one-third of its original
		speed, determine the mass of ball 2.
	}\\	
	\par
	\paragraph{39.}
	\textit{
		Suppose a watermelon with a mass of 2.0 kg undergoes
		a head-on elastic collision on a frictionless counter
		with a grapefruit with a mass of 0.8 kg. If the total
		kinetic energy of the system is 10.5 J and the total
		momentum is 7.5 kg·m/s, determine the possible
		initial and final velocities for the watermelon and
		the grapefruit
	}\\	
	\par
	\paragraph{49.}
	\textit{
		 A block of ice of mass 50.0 g slides along a
		frictionless, frozen lake at a speed of 0.30 m/s.
		It collides with a 100.0 g block of ice that is sliding
		in the same direction at 0.25 m/s (Figure	4). Th e two
		blocks stick together.
	}\\\\	
		\indent\textit{
		\textbf{A.} How fast are the two blocks moving aft er the
		collision? 
		}\\\\
		\indent\textit{
		\textbf{B.} How much kinetic energy is lost?
		}\\
	\par
	\paragraph{51.}
	\textit{
		Two equal-mass hockey pucks undergo a glancing
		collision. Puck 1 is initially at rest and is struck
		by puck 2 travelling at a velocity of 13 m/s [E].
		Puck 1 travels at an angle of [E 18° N] aft er the
		collision. Puck 2 travels at an angle of [E 4° S].
		Determine the fi nal velocity of each puck
	}\\	
	\par
	
\end{document}
